% Introduction

% Main chapter title
%\chapter[toc version]{doc version}
\chapter{Conclusion}

% Short version of the title for the header
%\chaptermark{version for header}

% Chapter Label
% For referencing this chapter elsewhere, use \ref{ChapterTemplate}
\label{Conclusion}

The proposed approach presents a novel application of \ac{mcts} to \ac{ucttp}, combined with \ac{hc} for local improvements. While \ac{mcts} has seen limited application in \ac{ucttp}, this dissertation demonstrates its potential.

The integration of \ac{hc} significantly improves solution quality by exploiting feasible regions more effectively.

An additional innovation is the incorporation of a diving strategy into the \ac{mcts} process. This mechanism allows the algorithm to focus more deeply on promising branches of the search tree, accelerating convergence and reducing unnecessary exploration.

However, our findings indicate that both standard and diving-based \ac{mcts} tend to plateau after a certain number of iterations, with diminishing returns even under extended run times. While diving produces more consistent improvements and tends to stay closer to the best-found solutions, the problem of long-term stagnation remains and represents a key area for future optimization.

Our experimental results, including comparisons against the benchmark dataset from the ITC-2007 competition, show that the hybrid \ac{mcts}-\ac{hc} approach consistently produces feasible solutions. Although it does not yet outperform state-of-the-art solutions, this dissertation provides a more effective and adaptive scheduling process for \ac{fcup} and can be extended to other institutions and help in other studies. 

Future work should focus on refining the diving strategy, refining the heuristic functions and improving computational performance.


