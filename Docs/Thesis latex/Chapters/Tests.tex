\pretocmd{\chapter}{\glsresetall}{}{}

% Tests

% Main chapter title
%\chapter[toc version]{doc version}
\chapter{Tests}


\label{Tests}

This chapter outlines the numerous tests used to evaluate the performance of the final algorithm on instances from the\ac{itc-2007} benchmark. %\footnote{https://www.eeecs.qub.ac.uk/itc2007/Login/SecretPage.php}. %All the tests were executed on a computer with the following characteristics: (...)

\section{Test Instances}

The evaluation used all 21 instances from the\ac{itc-2007} track 3 dataset (\texttt{comp01.ctt} to \texttt{comp21.ctt}). Each instance includes detailed information about about curricula, courses, rooms, unavailability constraints, lectures, periods and lecturers. The key features of each instance are summarized in Table~\ref{itc_instance_features}.

\begin{table}[h!]
\footnotesize
\begin{center}
\begin{tabular}{ |c|c|c|c|c|c|c|c| }
\hline
\textbf{Instance} & \textbf{Curricula} & \textbf{Courses} & \textbf{Rooms} & \textbf{Constraints} & \textbf{Lectures} & \textbf{Periods} & \textbf{Lecturers} \\
\hline
comp01                                    & 14                                         & 30                                       & 6                                      & 53                                           & 160                                       & 30                                       & 24                                        \\
comp02                                    & 70                                         & 82                                       & 16                                     & 513                                          & 283                                       & 25                                       & 71                                     \\
comp03                                    & 68                                         & 72                                       & 16                                     & 382                                          & 251                                       & 25                                       & 61                                        \\
comp04                                    & 57                                         & 79                                       & 18                                     & 396                                          & 286                                       & 25                                       & 70                                        \\
comp05                                    & 139                                        & 54                                       & 9                                      & 771                                          & 152                                       & 36                                       & 47                                        \\
comp06                                    & 70                                         & 108                                      & 18                                     & 632                                          & 361                                       & 25                                       & 87                                        \\
comp07                                    & 77                                         & 131                                      & 20                                     & 667                                          & 434                                       & 25                                       & 99                                        \\
comp08                                    & 61                                         & 86                                       & 18                                     & 478                                          & 324                                       & 25                                       & 76                                        \\
comp09                                    & 75                                         & 76                                       & 18                                     & 405                                          & 279                                       & 25                                       & 68                                        \\
comp10                                    & 67                                         & 115                                      & 18                                     & 694                                          & 370                                       & 25                                       & 88                                        \\
comp11                                    & 13                                         & 30                                       & 5                                      & 94                                           & 162                                       & 45                                       & 24                                        \\
comp12                                    & 150                                        & 88                                       & 11                                     & 1368                                         & 218                                       & 36                                       & 74                                        \\
comp13                                    & 66                                         & 82                                       & 19                                     & 468                                          & 308                                       & 25                                       & 77                                        \\
comp14                                    & 60                                         & 85                                       & 17                                     & 486                                          & 275                                       & 25                                       & 68                                        \\
comp15                                    & 68                                         & 72                                       & 16                                     & 382                                          & 251                                       & 25                                       & 61                                        \\
comp16                                    & 71                                         & 108                                      & 20                                     & 518                                          & 366                                       & 25                                       & 89                                        \\
comp17                                    & 70                                         & 99                                       & 17                                     & 548                                          & 339                                       & 25                                       & 80                                        \\
comp18                                    & 52                                         & 47                                       & 9                                      & 594                                          & 138                                       & 36                                       & 47                                        \\
comp19                                    & 66                                         & 74                                       & 16                                     & 475                                          & 277                                       & 25                                       & 66                                        \\
comp20                                    & 78                                         & 121                                      & 19                                     & 691                                          & 390                                       & 25                                       & 95                                        \\
comp21                                    & 78                                         & 94                                       & 18                                     & 463                                          & 327                                       & 25                                       & 76 					\\
\hline                                      
\end{tabular}
\caption{Key features of the\ac{itc-2007} track 3 instances.}
\label{itc_instance_features}
\end{center}
\end{table}

All instances guarantee at least one feasible solution, which means that in the optimum there are no violations of hard constraints. Additionally, to validate the correctness of a solution, the competition organizers provided a C++ source code that identifies the type of violation and calculates the corresponding penalty.

\section{Test Setup}

The tests were conducted for different values of the parameter \(C\) (Equation \ref{uct_formula}), which affects the exploration-exploitation trade-off in the\ac{mcts} algorithm. A higher \(C\) value increases the weight of the second term, allowing the algorithm explore less-visited nodes with more intensity. On the other hand, a lower \(C\) value favors nodes with a higher average reward, resulting in increased exploitation of known favorable choices. Specifically, we tested the algorithm with some values of C ranging from 0.1 to 1000 (0.1, 0.2, 0.5, 1, 1.4, 2, 5, 10, 20, 50, 100, 200, 500, 1000).

Additionally, an alternative\ac{uct} formula was evaluated, modifying the exploitation term to use the accumulated reward instead of the average (Equation \ref{modified_uct}), giving an advantage to the nodes that were exploited first.

\begin{equation}
UCT = w_i + 2C\sqrt{\frac{2\ln{n}}{n_i}},\label{modified_uct}
\end{equation} where \(w_i\) is the total reward of all playouts through this state, \(n_i\) is the number of visits of child node \(i\), \(C\) is a constant (typically \(\sqrt{2}\) in game-playing scenarios), and \(n\) is the number of visits of the parent node.

For optimization problems, the value of \(C\) is problem-dependent and usually needs to be empirically tuned rather than set to \(\sqrt{2}\) \cite{swiechowski_monte_carlo_review}.

In the\ac{hc} algorithm, the parameter \(HC\_IDLE\) was set to 5000, indicating the maximum number of consecutive iterations allowed without improvement before the search is terminated.

\section{Performance Metrics}

Each test corresponds, therefore, to a different \(C\) value, and the performance measurements are recorded for 21 problem instances, denoted as \texttt{comp01} to \texttt{comp21}. For each instance, the method iterates numerous times until a stopping condition is met, which is in this experiment a time constraint.

For each test, the following key performance metrics were considered:

\begin{itemize}
\item \textbf{Best hard penalty:} The lowest number of hard constraint violations found during the execution of the algorithm for a given instance. A value of zero indicates a feasible solution that satisfies all hard constraints.

\item \textbf{Worst hard penalty:} The highest number of hard constraint violations found during the execution of the algorithm for a given instance.

\item \textbf{Best soft penalty:} The lowest soft constraint penalty achieved during the search process for a given instance.

\item \textbf{Worst soft penalty:} The highest soft constraint penalty achieved during the search process for a given instance.

\item \textbf{Iteration of best solution:} The specific iteration at which the best solution (in terms of hard or soft penalties) was found.

\item \textbf{Total number of iterations:} The overall count of iterations performed during the algorithm's execution for each problem instance.

\item \textbf{Time to best solution:} The elapsed time required to reach the best solution for a given instance.
\end{itemize}


\section{Testing Procedure}

To thoroughly evaluate the algorithm's robustness and consistency, two different testing approaches were employed:

\begin{enumerate}

\item \textbf{1-hour and 24-hours runs:} The algorithm was executed for 1 hour and 24-hours to assess its ability to converge towards high-quality solutions given sufficient runtime.

\item \textbf{10-minute runs:} The algorithm was run for 10 minutes using ten different random generator seeds (1-10). Using fixed seeds ensured that the results could be reproduced, allowing for a fair comparison of different parameter settings and configurations. This approach also helped to evaluate the consistency and variability of results across different initializations. 

\end{enumrate}