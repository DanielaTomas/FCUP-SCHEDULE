% Introduction

% Main chapter title
%\chapter[toc version]{doc version}
\chapter{Introduction}

% Short version of the title for the header
%\chaptermark{version for header}

% Chapter Label
% For referencing this chapter elsewhere, use \ref{ChapterTemplate}
\label{Introduction}

% Write text in here
% Use \subsection and \subsubsection to organize text

\ac{ucttp} is a complex combinatorial optimization problem that consists of allocating events, rooms, lecturers, and students to weekly schedules while meeting certain constraints. A particular focus of this research is on \ac{cb-ctt}, a variant of \ac{ucttp} where the scheduling process is centered around courses and their associated curricula. 

%To illustrate the complexity, consider two events of the same course, "Introduction to Programming" and "Calculus I", both with weekly theoretical and practical lectures. "Introduction to Programming" may require a medium-sized lecture auditorium with multimedia facilities for the theoretical lectures and a computer laboratory for the practical lectures, while the second course, "Calculus I," may require a larger auditorium to accommodate more students for both types of lectures. These courses often have additional requirements, such as ensuring a lecturer is not scheduled for two courses simultaneously or that students enrolled in both courses have non-overlapping schedules. 

To illustrate the complexity of the problem, consider two courses within the same curricula: "Introduction to Programming" and "Calculus I." Each course comes with distinct scheduling demands: the former might require three lectures per week (two theoretical and one lab), while "Calculus I" may demand two lectures (a theoretical and a practical). Each lecture requires specific resources and teaching environments: practical programming lectures typically require a computer lab, while large auditoriums are better suited for theoretical lectures to accommodate more students. Since these lectures cannot overlap, variations in lecture frequency and format add an additional layer of complexity to the scheduling process. %Additional constraints further complicate the task, such as respecting lecturer availability and ensuring that students have balanced daily schedules without long gaps or excessive workloads. 
Moreover, other specific requirements may arise depending on the institution’s policies and the particular needs of courses or lecturers.

The challenge intensifies when considering a complete bachelor’s or master’s degree, which spans several years and involves numerous courses, each with multiple lectures and unique requirements. Given the scale and complexity of the problem, obtaining an optimal solution in usable time is typically infeasible. Nevertheless, heuristic algorithms have proved capable of producing approximate and good-quality solutions effectively. 

\section{Motivation and Objectives}

The process of building timetables at \ac{fcup} each semester is currently time-consuming, not automated, and the final results are not always the most satisfactory. In general, existing tools focus primarily on visualizing timetables or on basic conflict detection. This situation not only results in suboptimal schedules but also increases the workload for administrative staff and impacts the satisfaction of both lecturers and students. 

In the academic year 2021/2022, \ac{fcup} enrolled more than 4300 students, offered 127 different curricula, and coordinated 589 collaborators \cite{fcup_em_numeros}. This scale underscores the need for an effective timetabling system that can manage such scale while minimizing errors and conflicts. %The approach in use is not only inefficient but also prone to errors and conflicts. 

Therefore, the primary objective of this dissertation is to enhance the efficiency and quality of \ac{fcup}'s weekly timetable development by providing step-by-step interactive recommendations and detecting potential conflicts during its construction. This functionality must be integrated into a timetable visualization interface that was previously developed using reactive programming.

\section{Methodology and Contributions}

To address the problem, this dissertation employs a novel hybrid approach combining two heuristic algorithms: \ac{mcts} and \ac{hc}. The \ac{mcts} heuristic search algorithm was chosen, as it has been applied to various optimization problems and has proven to be particularly effective in games. \ac{hc} was also chosen to be used in conjunction with \ac{mcts} for local optimization. Although \ac{mcts} has shown positive results in relevant areas, its application to \ac{ucttp} remains largely unexplored, presenting an opportunity to advance the state of the art and help in further studies. 

The proposed system leverages the global search capabilities of \ac{mcts} to explore diverse scheduling possibilities and applies \ac{hc} to refine solutions locally. This hybrid approach aims to deliver better-quality timetables while addressing the practical challenges faced by \ac{fcup} in managing large-scale scheduling.

\section{Dissertation layout}

The remainder of the dissertation is organized as follows:

\begin{itemize}
\item \textbf{Chapter~\ref{Background} - Background:}
\item \textbf{Chapter~\ref{Related Work} -  Related Work:}
\item \textbf{Chapter~\ref{Development} -  Development:}
\item \textbf{Chapter~\ref{Tests} - Tests:}
\item \textbf{Chapter~\ref{Results} -  Results:}
\item \textbf{Chapter~\ref{Conclusion} - Conclusion:} Summarizes the findings, discusses the contributions, and outlines potential future enhancements.
\end{itemize}




%Details the design of the proposed hybrid optimization system, including algorithmic formulation and system architecture.
%Describes the implementation of the system, experimental setup, and benchmarking using ITC-2007 inspired datasets.
%Presents and analyzes the experimental results, comparing the proposed method against existing approaches.
%This chapter reviews existing timetabling methodologies and heuristic algorithms, with a focus on \ac{mcts} and local search techniques.



% To illustrate the complexity of the problem, consider two courses within the same curricula: "Introduction to Programming" and "Calculus I". Each course comes with distinct scheduling demands and logistical challenges. The former might require three lectures per week: two theoretical and one practical lab. In contrast, "Calculus I" may demand two lectures: a theoretical, and a practical. Each lecture need specific resources and teaching environments. Practical programming lectures typically require computer lab, while large auditoriums are better suited for theoretical lectures to accommodate more students. Since these classes cannot overlap, variations in session frequency and format add an additional layer of complexity to the scheduling process. Further complicating the task are additional constraints, such as respecting lecturer availability and ensuring that students have balanced daily schedules without long gaps or excessive workloads. Moreover, other specific requirements may arise depending on the institution’s policies and the particular needs of courses or lecturers, further complicating the timetable construction.

%O Capítulo~\ref{} aborda conceitos que são necessários para uma melhor contextualização e compreensão do problema em questão e dos algoritmos utilizados para obter o resultado final.