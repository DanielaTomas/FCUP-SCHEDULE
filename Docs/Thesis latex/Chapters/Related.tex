% Introduction

% Main chapter title
%\chapter[toc version]{doc version}
\chapter{Related Work}

% Short version of the title for the header
%\chaptermark{version for header}

% Chapter Label
% For referencing this chapter elsewhere, use \ref{ChapterTemplate}
\label{Related Work}

The literature on the \ac{ucttp} is vast, so this chapter will cover various approaches to address this problem, mostly based on surveys.

According to Lewis' survey \cite{lewis_survey_2008}, algorithms can be divided into three categories:

\begin{itemize}
	\item \textbf{One-Stage} algorithms use a weighted sum function of constraints to identify solutions that satisfy both hard and soft constraints at the same time;
	\item \textbf{Multi-Stage} algorithms aim to optimize soft constraints while ensuring feasibility;
	\item \textbf{Multi-stage with relaxation} is divided into multiple stages, with some constraints being relaxed while satisfying others.
\end{itemize}

Other surveys, such as Abdipoor et al. \cite{abdipoor_meta-heuristic_2023}, categorized \ac{ucttp} into five groups: operational research (or), metaheuristics, hyperheuristics, multi-objective, and hybrid approaches.

\section{Operational Research (OR)}

Operational Research (OR) techniques provide mathematically rigorous solutions. Despite their complexity, these methods are simple to implement and can be considered if time and memory constraints are not a concern \cite{babaei_survey_2015}. However, the other types of approaches provide better quality results. 

As Chen et al. and  Babaei et. al detailed in their surveys \cite{babaei_survey_2015, chen_survey_2021}, OR includes Graph Coloring (GC), Integer and Linear Programming (IP/LP) and Constraint Satisfaction(s) Programming (CSPs) based techniques.

\section{Metaheuristics}

Metaheuristics \textit{provide "acceptable" solutions in a reasonable time for solving hard and complex problems} \cite{talbi2009metaheuristics}. 

Metaheuristics are similarly defined by Du et al. as \textit{a class of intelligent self-learning algorithms for finding near-optimum solutions to hard optimization problems} \cite{du2016search}.

Metaheuristics, while not guaranteeing global optimality, have become one of the most used solution strategy for \ac{ucttp} and can be classified into two categories: Single solution-based and Population-based.

\subsection{Single Solution-Based}

Single solution-based metaheuristics, also known as local search algorithms, focus on modifying a single solution throughout the search in order to improve that solution. 

This metaheuristic approach includes algorithms such as Hill Climbing (HC), Simulated Annealing (SA), Tabu Search (TS), and Iterated Local Search (ILS). Simulated Annealing is perhaps the most effective single solution-based metaheuristic, particularly in benchmark datasets, but Tabu Search has also been shown to be successful in minimizing hard constraints \cite{abdipoor_meta-heuristic_2023}.

\subsection{Population-Based}

Population-based meta-heuristics iteratively improve a population of solutions by generating a new population in the same neighborhood as the existing ones. This method can be subdivided into Evolutionary Algorithms, such as Genetic Algorithms, and Swarm Intelligence, such as Ant Colony Optimization \cite{abdipoor_meta-heuristic_2023,du2016search}.

Population-based metaheuristics, as opposed to single-solution-based metaheuristics, are more focused on exploration rather than exploitation \cite{talbi2009metaheuristics,du2016search}. Despite promising results on real-world datasets, population-based approaches were rarely applied to benchmark datasets and performed poorly in \ac{itc} competitions compared to single solution-based methods \cite{abdipoor_meta-heuristic_2023}.

%\subsubsection{Hybrid}

\section{Hyperheuristics}

Heuristic methods have been highly effective in solving a wide range of problems. However, their application to new problems is often challenging due to the vast number of parameter tuning and the lack of clear guidance \cite{hyper_heuristics_survey}. To address this problem, hyperheuristics aim to generalize methods by selecting or combining the most suitable heuristic(s) for a specific problem, rather than explicitly solving the problem.

Hyper-heuristic and multi-objective approaches are less common, possibly due to their performance \cite{chen_survey_2021}.

\section{Multi-Objective}

Multi-objective or multi-criteria approaches aim to optimize multiple conflicting objectives simultaneously. These methods are frequently used to find the optimal Pareto front, which is a set of compromise optimal solutions. However, the disadvantage lies in the execution effort \cite{chen_survey_2021}. Several multi-criteria algorithms can also be included in other categories, such as metaheuristics.

\section{Hybrid}

Hybrid approaches mix algorithms from two or more of the previously mentioned types of approaches.

Hybrid approaches can indeed be divided into two main categories \cite{abdipoor_meta-heuristic_2023}:

\begin{itemize}
\item \textbf{Collaborative hybrids:} Involve running different algorithms sequentially, intertwined, or in parallel, sharing information without structural integration. This collaboration enables each component to focus on its area of the problem, with occasional communication guiding the overall search process.

\item \textbf{Integrative hybrids:} Integrate components of different algorithms into a single framework, creating a cohesive and interdependent method.

\end{itemize}

%Hybridization of local search and population-based approaches (also known as the Memetic Algorithm (MA))

In particular, Tomáš Müller, winner of the \ac{itc-2007} tracks 1 and 3, developed a collaborative hybrid approach\footnote{https://github.com/tomas-muller/cpsolver-itc2007} for all three tracks \cite{muller_itc2007}, finding feasible solutions for all instances of tracks 1 and 3 and most instances of track 2. 

The algorithms used included Iterative Forward Search, \ac{hc}, Great Deluge, and optionally Simulated Annealing. During the construction phase, Iterative Forward Search is employed to find a complete solution, followed by \ac{hc} to find the local optimum. When a solution cannot be improved further using \ac{hc}, Great Deluge is used to iteratively decrease, based on a cooling rate, a bound that is imposed on the value of the current solution. Optionally, Simulated Annealing can be used when the bound reaches its lower limit.

The solver's techniques have been integrated into UniTime\footnote{https://www.unitime.org/} system, which is widely used for scheduling in academic institutions. Furthermore, the principles behind Müller's hybrid methodology are still relevant in modern optimization problems.

While Müller's approach was highly successful in 2007, subsequent research has revealed that other methods may outperform it in some instances. However, it remains a benchmark against which new approaches are often compared.

\section{Monte Carlo Tree Search}

\ac{mcts} application to \acp{cop} remains relatively unexplored. In the context of \ac{ucttp}, Goh's investigation \cite{goh_MCTS} is the only known study that explores the potential of \ac{mcts} in solving the \ac{pe-ctt}. 

Goh's investigation employs a two-stage approach, first utilizing \ac{mcts} to find initial feasible solutions and then applying local search techniques to enhance the quality of these solutions.

The research introduces several enhancements to the standard \ac{mcts} algorithm, including heuristic-based simulations and tree pruning techniques, which improved the \ac{mcts} performance and effectiveness. The study also provides a comprehensive comparison of \ac{mcts} against traditional methods such as graph coloring heuristic and tabu search, with tabu search emerging as the most effective method.

One major limitation of \ac{mcts} for timetabling is, as highlighted by Goh, its rigorous decision-making process. Events are assigned sequentially and cannot be reassigned, limiting the algorithm's flexibility. In contrast, local search methods, such as tabu search, allow dynamic reassignment, enabling a more efficient exploration of the search space. Furthermore, the tree-based structure of \ac{mcts} restricts its hybridization with local search techniques, which have proven essential in achieving significant results in other learning-based algorithms. Another drawback of \ac{mcts} is its high computational cost, making it less effective under the strict time constraints imposed by timetabling problems.

This dissertation aims to demonstrate that, despite its limitations, \ac{mcts} still holds potential for timetabling problems with the appropriate modifications and hybridization. 





















