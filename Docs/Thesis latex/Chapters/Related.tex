% Introduction

% Main chapter title
%\chapter[toc version]{doc version}
\chapter{Related Work}

% Short version of the title for the header
%\chaptermark{version for header}

% Chapter Label
% For referencing this chapter elsewhere, use \ref{ChapterTemplate}
\label{Related Work}

The literature on the \ac{ucttp} is vast, so this chapter will cover various approaches to addressing this problem, mostly based on surveys.

According to Lewis' survey \cite{lewis_survey_2008}, algorithms can be divided into three categories:

\begin{itemize}
	\item \textbf{One-Stage} algorithms use a weighted sum function of constraints to identify solutions that satisfy both hard and soft constraints at the same time;
	\item \textbf{Multi-Stage} algorithms aim to optimize soft constraints while ensuring feasibility;
	\item \textbf{Multi-stage with relaxation} is divided into multiple stages, with some constraints being relaxed while satisfying others.
\end{itemize}

Other surveys, such as Abdipoor et al. \cite{abdipoor_meta-heuristic_2023}, categorized \ac{ucttp} into different categories, including operational research, metaheuristics, hyperheuristics, multi-objective, and hybrid approaches.

\section{Operational Research (OR)}

Operational research approaches, despite their complexity, are simple to implement and can be considered if time and space constraints are not a concern. However, the other types of approaches provide better results \cite{babaei_survey_2015}. As Chen et al. and  Babaei et. al detailed in their surveys \cite{babaei_survey_2015, chen_survey_2021}, OR includes Graph Coloring (GC), Integer and Linear Programming (IP/LP) and Constraint Satisfaction(s) Programming (CSPs) based techniques.

\section{Metaheuristics}

Metaheuristics \textit{provide "acceptable" solutions in a reasonable time for solving hard and complex problems} \cite{talbi2009metaheuristics}. 

Metaheuristics are similarly defined by Du et al. as \textit{a class of intelligent self-learning algorithms for finding near-optimum solutions to hard optimization problems} \cite{du2016search}.

In recent years, metaheuristics have become the most used solution strategy for \ac{ucttp} and can be classified into two categories: Single solution-based and Population-based.

\subsection{Single Solution-Based}

Single solution-based metaheuristics, also known as local search algorithms, focus on modifying a single solution throughout the search in order to improve that solution. This metaheuristic approach includes algorithms such as Hill Climbing (HC), Simulated Annealing (SA), Tabu Search (TS), and Iterated Local Search (ILS). Simulated Annealing is perhaps the most effective single solution-based metaheuristic, particularly in benchmark datasets, but Tabu Search has also been shown to be successful in minimizing hard constraints \cite{abdipoor_meta-heuristic_2023}.

\subsection{Population-Based}

Population-based meta-heuristics iteratively improve a population of solutions by generating a new population in the same neighborhood as the existing ones. This method can be subdivided into Evolutionary Algorithms, such as Generic Algorithms, and Swarm Intelligence, such as Ant Colony Optimization \cite{abdipoor_meta-heuristic_2023,du2016search}.

Population-based metaheuristics, as opposed to single-solution-based metaheuristics, are more focused on exploration rather than exploitation \cite{talbi2009metaheuristics,du2016search}. Despite promising results on real-world datasets, population-based approaches were rarely applied to benchmark datasets and performed poorly in \ac{itc} competitions compared to single solution-based methods \cite{abdipoor_meta-heuristic_2023}.

%\subsubsection{Hybrid}

\section{Hyperheuristics}



\section{Multi-Objective}

Multi-objective or multi-criteria approaches aim to optimize multiple conflicting objectives simultaneously. These methods are frequently used to find the optimal Pareto front, which is a set of compromise optimal solutions. However, the disadvantage lies in the execution effort \cite{chen_survey_2021}. Several multi-criteria algorithms can also be included in other categories, such as metaheuristics.

\section{Hybrid}

Hybrid approaches mix algorithms from two or more of the previously mentioned types of approaches.

%Hybridization of local search and population-based approaches (also known as the Memetic Algorithm (MA))

In particular, Tomáš Müller, winner of the \ac{itc-2007}\footnote{https://www.eeecs.qub.ac.uk/itc2007/} tracks 1 and 3, developed a hybrid approach for all three tracks \cite{muller_itc2007}, finding feasible solutions for all instances of tracks 1 and 3 and most instances of track 2. The algorithms used included Iterative Forward Search, Hill Climbing, Great Deluge, and optionally Simulated Annealing. During the construction phase, Iterative Forward Search is employed to find a complete solution, followed by Hill climbing to find the local optimum. When a solution cannot be improved further using Hill Climbing, Great Deluge is used to iteratively decrease, based on a cooling rate, a bound that is imposed on the value of the current solution. Optionally, Simulated Annealing can be used when the bound reaches its lower limit.

%Müller [34] solves the problem by applying a constraint-based solver that incorporates several local search algorithms operating in three stages: (i) a construction phase that uses an Iterative Forward Search algorithm to find a feasible solution, (ii) a first search phase delegated to a Hill Climbing algorithm, followed by (iii) a Great Deluge or Simulated Annealing strategy to escape from local minima. The algorithm was not specifically designed for CBCTT but it was intended to be employed on all three tracks of ITC2007 (including, besides CB-CTT and PE-CTT, also Examination Timetabling). The solver was the winner of two out of three competition tracks, and it was among the finalists in the third one